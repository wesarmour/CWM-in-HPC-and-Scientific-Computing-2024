\documentclass[a4paper, 12pt]{article}


%
% ----- packages
%
\usepackage {fancyhdr}
\usepackage[dvips]{epsfig}


%
% ----- settings
%
\addtolength{\evensidemargin}{-20mm}
\addtolength{\oddsidemargin}{-20mm}
\addtolength{\textwidth}{+34mm}
\addtolength{\topmargin}{-30mm}
\addtolength{\textheight}{+24mm}
\addtolength{\parskip}{+2mm}
\addtolength{\headsep}{2mm}
\setlength{\parindent}{0mm}


%
% ----- fancy header
%
\setlength  {\headheight}{16pt}
\pagestyle  {fancyplain}

%\lhead{\fancyplain{}{
%    \epsfig{figure=artwork/arc_logo_rgb_solid.eps, width=0.08\textwidth}
%}}
\rhead{}
%\rhead{\fancyplain{}{
%    \epsfig{figure=artwork/ant.eps,                width=0.08\textwidth}
%}}
%\lfoot{\fancyplain{}{\bfseries An Introduction to HPC and Scientific Computing -- Compiling and Batch Systems}}
\rfoot{\fancyplain{}{\bfseries\thepage}}


%%%%% of \pageref{LastPage}}}

\renewcommand{\headrulewidth}{0.2pt}
\renewcommand{\footrulewidth}{0.2pt}
\cfoot{}


%
% ----- symbols
%

\def \tld  {$\sim$}

\def \cc   {\tt }               % computer code
\def \ccem {\tt\bf }            % computer code emphasized
\def \eg   {{\em e.g.\ }}
\def \ie   {{\em i.e.\ }}

%
% ----- part
%
\newcommand{\articlesection}[1]{%
  \thispagestyle{plain}
  \vspace*{0.4\textheight}
  \begin{center}
    {\Huge\bf {#1}}
  \end{center}
  \newpage
}


%====================================================================%
\title{{\Huge\bf An Introduction to HPC and Scientific Computing -- Compiling and Batch Systems\footnotetext{Thanks to Jacob Wilkins and Ian Bush for materials.}} \\ {\huge -- Practical Sessions --}}

\date{}

%====================================================================%

\begin{document}

\maketitle

\vfill

\tableofcontents

\newpage


\section{Introduction}
%        ==========
\label{Introduction}

In these exercises we will look at compiling C programs, and introduce use of the batch system.

The practicals will be performed on the cluster. Log on using the username and password that have been provided, and the files are under the directory {\cc notes}

Remember, to log in use: \\
{\cc ssh -CX [username]@htc-login.arc.ox.ac.uk}

Load git module: \\
{\cc module load git}

Clone CWM repository if you not already done so: \\
{\cc  git clone https://github.com/wesarmour/CWM-in-HPC-and-Scientific-Computing-2024.git}



%                                                                       %
% ===================================================================== %
%                                                                       %
%                                                                       %
%             P A R T   1   --   Hello World                            %
%                                                                       %
%                                                                       %
% ===================================================================== %
%                                                                       %

\section{Hello World}
%        ==========
\label{basics}

\subsection*{About}
%            -----

This exercise introduces compiling with
one of the simplest programs possible, the ``hello world'' code.  The source
files are found in the directory {\cc hello}:
%
\begin{itemize}
  \item {\cc hello\_world.c} -- ``hello world'' program 
\end{itemize}
%

\subsection*{Tasks}
%            -----

\begin{enumerate}

  \item Before we start each question we need to set up the software 
    environment. First issue a {\cc module purge} command to ensure
    that you are stating from a clean sheet. Now we need to load
    the module that makes the compiler available.
    Throughout these exercises we shall use Intel compiler. To load this issue 
    {\cc module load intel-compilers}. If you have time at the end
    you might to try the practicals with the gnu
    compiler. If so instead use {\cc module load gcc binutils}.
    You won't need to change any of your programs if you do this, but the
    makefiles will need changing as the gnu compiler accepts different flags, and uses
    the command gcc to invoke the compiler.
  \item Change directory to the location of the code for this first practical:\\
{\cc cd CWM-in-HPC-and-Scientific-Computing-2024/Practicals/prac1/code/hello/}

  \item Edit the ``hello world'' program {\cc hello\_world.c} and try to understand it

  \item Compile the program with {\cc icc hello\_world.c}

  \item Use \texttt{ls} to see executable which the compilation has made, {\cc a.out} in this case

  \item Run the executable. Does it give the output you expect?

  \item Re-compile, this time with an appropriate flag so the executable is called {\cc hello}
    Check you have done this successfully via ls, and running the new executable

\end{enumerate}


%                                                                       %
% ===================================================================== %
%                                                                       %
%                                                                       %
%             P A R T   2   --   Makefiles                              %
%                                                                       %
%                                                                       %
% ===================================================================== %
%                                                                       %

\section{Makefiles}
%        ===========================
\label{integral}


\subsection*{About}
%            -----

In this exercise we introduce compiling a program which is held in multiple files, and also the
use of a makefile. The example implements two different ways to orthonormalise a set of vectors,
i.e. adjust the vectors so that they are all at right angles to each other, and compares the time
for the two methods. Whilst rather abstract this is actually part of a number of algorithms commonly
used in computational science.

The source files are found in the directory {\cc gs}:
%
\begin{itemize}
  \item {\cc gs\_opt.c} -- the main code,
  \item {\cc timer.c} -- routines that implement the timing,
  \item {\cc array\_alloc.c} -- routines that implement memory allocation
\end{itemize}
%

\subsection*{Tasks}
%            -----

\begin{enumerate}

   \item It's good practice to make sure that you have a clean
    software environment at the start of each question. So
    again issue {\cc module purge} and
    {\cc module load intel-compilers}. Are you sure you
    understand what these commands are doing?

  \item Compile the code with {\cc icc gs\_opt.c array\_alloc.c timer.c -o gs}. Run the program
    and enter 1000 for both the order and number of the vectors and look at the output

  \item This command is getting a bit long already with just 3 files! We have also provided
    a makefile to help you compile this program. Recompile the program by issuing
    the command {\cc make} and run the program as before

  \item Issue an {\cc ls} command - can you see the object files?

  \item Issue the command {\cc touch timer.c} and then {\cc make}. What happens? How is it
    different from the previous {\cc make}?

  \item Run the program again with the input as before. Note down the times taken for the two methods.

  \item Open up the makefile with your editor of choice. In it you will see a line looking like

    {\cc CFLAGS = -O0}

    This controls the flags used by the compiler. In particular the {\cc -O} flag controls how hard the compiler
    works to make an executable that runs quickly. This is called ``optimisation''. A small number, such as zero, means
    little effort is put in. A large number (the maximum is typically 3) means a lot of work is put in. Change this line to read

    {\cc CFLAGS = -O2}

    recompile the program using {\cc make clean} and then {\cc make} and re-run it. How have the times changed?

  \item try {\cc -O1} and {\cc -O3}, how does this affect the run time? Remember to use {\cc make clean} between each recompile to make
    sure everything is rebuilt from scratch.

\end{enumerate}


%                                                                       %
% ===================================================================== %
%                                                                       %

\section{Batch Systems}
%        ========================
\label{heat}

\subsection*{About}
%            -----
We shouldn't really run any serious jobs on the front end nodes. As you have to share the front end nodes with
other users the timings we have gathered so far are not that reliable, also they may impede other people's work and get you angry emails from the systems administrator.
The correct way is to use the compute nodes via
the batch system. In the {\cc gs} directory you should have also noticed a file called {\cc script.sl}, this is an appropriate
batch script for the {\cc htc-login} system and we will now use it to run the {\cc gs} program on the compute nodes

\subsection*{Tasks}
%           ------

\begin{enumerate}
  
  \item Examine the {\cc script.sl} with an editor. Do you understand what it does? Think carefully about how it reserves the
     resources we need to run the program, and also about how the input is provided for the program - on the compute node there won't be a keyboard attached to provide that input!
  
  \item Issue the command {\cc sbatch script.sl}. What happens?

  \item Examine the running job with {\cc squeue -u <username>} replacing username as appropriate. Note the the number of the job is the same as that printed out by the {\cc sbatch} command

  \item When the job is finished use {\cc ls} to find the output. You should find a new file in the directory called something
     similar to {\cc slurm-1702775.out}, where the number will be the same as that printed out by sbatch. Examine the contents of this file.

  \item Re-run the timing exercise from the previous question where you examined how fast the program ran
     as a function of optimisation level, and see if the time in batch differs much from that measured on the front end

  \item Rather than {\cc slurm-1702775.out} how can you get the output written to a filename of your choice? \\
  Slurm's {\cc sbatch} docs might be useful {\cc https://slurm.schedmd.com/sbatch.html}

\end{enumerate}

% %                                                                       %
% % ===================================================================== %
% %                                                                       %
\end{document}

% end