
\documentclass[a4paper, 12pt]{article}


%
% ----- packages
%
\usepackage {fancyhdr}
\usepackage[dvips]{epsfig}


%
% ----- settings
%
\addtolength{\evensidemargin}{-20mm}
\addtolength{\oddsidemargin}{-20mm}
\addtolength{\textwidth}{+34mm}
\addtolength{\topmargin}{-30mm}
\addtolength{\textheight}{+24mm}
\addtolength{\parskip}{+2mm}
\addtolength{\headsep}{2mm}
\setlength{\parindent}{0mm}


%
% ----- fancy header
%
\setlength  {\headheight}{16pt}
\pagestyle  {fancyplain}

%\lhead{\fancyplain{}{
%    \epsfig{figure=artwork/arc_logo_rgb_solid.eps, width=0.08\textwidth}
%}}
\rhead{}
%\rhead{\fancyplain{}{
%    \epsfig{figure=artwork/ant.eps,                width=0.08\textwidth}
%}}
%\lfoot{\fancyplain{}{\bfseries An Introduction to HPC and Scientific Computing -- How to multi-task on CPUs using OpenMP}}
\rfoot{\fancyplain{}{\bfseries\thepage}}


%%%%% of \pageref{LastPage}}}

\renewcommand{\headrulewidth}{0.2pt}
\renewcommand{\footrulewidth}{0.2pt}
\cfoot{}


%
% ----- symbols
%

\def \tld  {$\sim$}

\def \cc   {\tt }               % computer code
\def \ccem {\tt\bf }            % computer code emphasized
\def \eg   {{\em e.g.\ }}
\def \ie   {{\em i.e.\ }}

%
% ----- part
%
\newcommand{\articlesection}[1]{%
  \thispagestyle{plain}
  \vspace*{0.4\textheight}
  \begin{center}
    {\Huge\bf {#1}}
  \end{center}
  \newpage
}


%====================================================================%
\title{{\Huge\bf An Introduction to HPC and Scientific Computing -- How to Multi-task on CPUs Using OpenMP} \\ {\huge -- Practical Sessions --}}

\date{}

\author{\bf
  \begin{tabular}{ll}
    Ian Bush   & {\em ian.bush@oerc.ox.ac.uk}
  \end{tabular}
}
%====================================================================%

\begin{document}

\maketitle

\vfill

\tableofcontents

\newpage


\section{Introduction}
%        ==========
\label{Introduction}

In these exercises we will look at programming with OpenMP. When answering them firstly
remember to always check that any programs you have written are correct, it doesn't matter
how fast it runs or how well it scales if the answers are wrong! Secondly in some of the
questions you are asked to look at the performance of the code. To do this you really
must do this with the batch system (we provide the batch scripts) as running on the front
end will not give reliable timings due to the interaction with other users.

The practicals will be performed on arcus-htc. Log on using the username and password that have been provided,
and the files are under the directory {\cc lecture6}

Remember, to log in use: \\
{\cc ssh -CX [username]@oscgate.arc.ox.ac.uk}

{\cc ssh -CX arcus-htc}


%                                                                       %
% ===================================================================== %
%                                                                       %
%                                                                       %
%             P A R T   1   --   Hello World                            %
%                                                                       %
%                                                                       %
% ===================================================================== %
%                                                                       %

\section{Hello World}
%        ==========
\label{basics}

\subsection*{About}
%            -----

This exercise introduces the basic functionality of OpenMP with
one of the simplest programs possible, the ``hello world'' code.  The source
files are found in the directory {\cc hello}:
%
\begin{itemize}
  \item {\cc hello\_omp.c} -- the OpenMP ``hello world'' program 
\end{itemize}
%

\subsection*{Tasks}
%            -----

\begin{enumerate}

  \item Before we start each question we need to set up the software 
    environment. First issue a {\cc module purge} command to ensure
    that you are stating from a clean sheet. Now we need to load
    the module that makes the compiler available.
    Throughout these exercises we shall use gnu compiler. To load this issue 
    {\cc module load gcc}. If you have time at the end
    you might to try the practicals with the intel
    compiler. If so instead use {\cc module load intel-compilers}.
    You won't need to change any of your programs if you do this, but the
    makefiles will need changing as the intel compiler accepts different flags.

  \item Edit the ``hello world'' program {\cc hello\_omp.c} and try to understand it

  \item Edit the file {\cc makefile} and try to figure out what it
    does.  Compile the OpenMP ``hello world'' programs by
    running the command {\cc make}.

  \item Run the executable.  To do this, submit the provided job
    script {\cc run\_hello\_omp.sh} with the command {\cc sbatch}.  Observe the
    output, which will be in a file called like {\cc slurmXXXXX.out} created by the batch environment in the directory that you submitted the job, and try to correlate that with the source code.

  \item Run the executable several times.  Does the order in which
    each thread outputs a message change?

  \item Repeat the exercise but using just 4 threads. How do you achieve this?

\end{enumerate}


%                                                                       %
% ===================================================================== %
%                                                                       %
%                                                                       %
%             P A R T   2   --   Integration                            %
%                                                                       %
%                                                                       %
% ===================================================================== %
%                                                                       %

\section{Trapezoidal Integration}
%        ===========================
\label{integral}


\subsection*{About}
%            -----

This exercise is based on a more ``real-life'' application than the
previous: the computation of a function integral using the trapezoidal
rule.

The source files are found in the directory {\cc integral}:
%
\begin{itemize}
  \item {\cc integral.c} -- a serial implementation,
  \item {\cc integral\_omp.c} -- the OpenMP implementation
\end{itemize}
%

You are asked to complete the programming of the OpenMP source
code, starting from the existing files. The OpenMP code uses parallel region
construct and the OpenMP reduction operation. 

\subsection*{Notes}
%            -----

\begin{itemize}

  \item The serial implementations use the function {\cc trapInt},
    defined in file {\cc integral}.  With two real variables ({\cc a}
    and {\cc b}) and an integer ({\cc N}) as arguments, this function
    computes the integral of the function {\cc f()} (also defined in
    {\cc integral}) between {\cc a} and {\cc b}.  {\cc trapInt} uses
    the simple trapezoidal rule on {\cc N} trapeziums to compute the
    integral of {\cc f()}.
    
  \item All code is programmed to read the parameters {\cc a}, {\cc b}
    and {\cc N} from standard input.  The standard input can be
    redirected from a file, such that the exectables can be run in a
    non-interactive manner.  The input file is provided; to run the
    serial implementation, for example, the command is {\cc integral <
      integral.inp}.

  \item All source code uses real numbers in double precision.

  \item Time is measured in the serial code {\cc integral} using the
    function {\cc wall\_clock\_time} (a wrapper around {\cc
      gettimeofday}). In the OpenMP code, it is replaced
    with the OpenMP function {\cc omp\_get\_wtime}, although {\cc
      wall\_clock\_time} would work just as well.
\end{itemize}


\subsection*{Tasks}
%            -----

\begin{enumerate}

  \item It's good practice to make sure that you have a clean
    software environment at the start of each question. So
    again issue {\cc module purge} and
    {\cc module load gcc}. Are you sure you
    understand what these commands are doing?

  \item Compile the serial code ({\cc make integral}) and run it with
    the command {\cc ./integral < integral.inp}, using the input file
    provided {\cc integral.inp}.  Using equal values for {\cc a} and
    {\cc b} but opposite in sign and {\cc N=1024}, the integral value
    is approximately pi. This is correct as the integral that is being
    evaluated is {\cc f(x)=2.0*sqrt(1.0-x*x)}, and the bounds are -1 to 1;
    the analytic answer is exactly pi.

  \item Edit {\cc integral\_omp} and program the threaded computation
    of the integral.  To achieve this simply transform the original
    serial {\cc trapInt} function by threading the {\cc for} loop and
    use the threaded function to compute the integral.  Remember that
    the OpenMP directive must include a reduction on the integral
    value variable {\cc v}.  Compile the OpenMP code with the command
    {\cc make integral\_omp}, run it using the script {\cc
        run\_integral\_omp.sh} provided and verify that it produces the
    correct answer: {\cc sbatch run\_integral\_omp.sh}.

  \item Look at how the time to solution for your OpenMP program varies
    with the number of intervals being used (the last number in the input file),
    and also the number of threads. Examine 10000, 100000 and 1000000 intervals,
    and 1, 2, 4, 8 and 16 threads. Plot a speed up curve for your results -
    Speed up is how much faster a calculation runs on P threads compared to the same
    calculation on 1 thread, i.e. {\cc S=time(1)/time(P)}. Can you make sense of
    these results?

\end{enumerate}


%                                                                       %
% ===================================================================== %
%                                                                       %

\section{The 1D Heat Equation}
%        ========================
\label{heat}

\subsection*{About}
%            -----

The code in this exercise solves the one-dimensional heat equation
using finite differences and OpenMP programming.  The
finite-difference method provides a realistic yet simple algorithmic
framework to exercise some of the important basic aspects of parallel
programming: communication, synchronisation and reduction.

The source files are found in the directory {\cc heat}.  The source files are:
%
\begin{itemize}
  \item {\cc heat.c} -- a serial implementation
  \item {\cc heat\_omp.c} -- the corresponding OpenMP implementation
  \item {\cc heat\_omp\_easy.c} -- the ``easy'' OpenMP implementation
\end{itemize}


\subsection*{Notes}
%            -----

Without going deep into the numerical analysis aspects, here are the
key concepts to help you go though this exercise.

The 1D heat equation is a second order partial differential equation,
whose solution describes the time-evolution of temperature along a
heat-conducting rod, starting from an initial prescribed temperature
distribution.  In this exercise, it is considered that the
temperatures at both ends (the boundary conditions) of the rod are
constant (Dirichlet type) and equal to zero.  Mathematically, we seek
the solution to
%
\begin{eqnarray}
  && \frac{\partial u}{\partial t} = \frac{\partial^2 u}{\partial t^2} \\
  && u(x, 0) = u_0(x) \\
  && u(0, t) = u(1, t) = 0
\end{eqnarray}

The numerical solution to this problem consists of an array of unknown
temperature values corresponding to equally spaced points along the
rod at any given time.  The numerical algorithm chosen for this
exercise updates this array within a discrete time-loop, with all the
temperature values at one time-level depending on the temperature at
the previous time-level (explicit scheme).  The outcome of the
algorithm is thus the temperature distribution at the final simulated
time.

Assuming the rod extends from 0 to 1, and {\cc n} is the number of
points along the rod (corresponding to {\cc n-1} equal-length
intervals), the coordinate of the points are {\cc x[j] = j*dx} with
{\cc dx = 1/(n-1)} and {\cc j} going from {\cc 0} to {\cc n-1}.
Also, denoting by {\cc n\_time\_steps} the number of
time levels for which the solution is computed and by {\cc u} the
array of temperature values, the numerical algorithm is the following
time for-loop in C
%
\begin{verbatim}
        // time loop
        for (t=0; t<n_time_steps; t++) {

          // store the old solution
          for (j=1; j<n-1; j++) {
            uo[j] = u[j];
          }

          // finite difference scheme
          for (j=1; j<n-1; j++) {
            u[j] = uo[j] + nu*(uo[j-1]-2.0*uo[j]+uo[j+1]);
          }
        }
\end{verbatim}
%

The above code snippet represent the core of the serial source code {\cc heat}.

Note how at each time step {\cc n} and before the update, the current
temperature values in {\cc u} are copied to an additional array {\cc
  uo}, from which they are used within the update.  (A pointer swap is
obviously more efficient for achieving this but less clear in an
exercise.)

The key part of this algorithm is the C line
%
\begin{verbatim}
        u[j] = uo[j] + nu*(uo[j-1]-2.0*uo[j]+uo[j+1]);
\end{verbatim}
%

in which {\cc nu} is a constant that has to be less than or equal to 0.5 in
order to preserve numerical stability of the scheme.  The array values
{\cc u} are updated based on the values of the previous time-step,
stored in {\cc uo}.  At each point {\cc j} along the rod, the update
is carried out in a stencil involving three points along the rod ({\cc
  j-1}, {\cc j} itself and {\cc j+1} as well as two points along the
time-line (again {\cc j} but for time steps {\cc t} and {\cc t-1}).

\subsection*{Tasks}
%           ------

\begin{enumerate}
  
  \item Continue practicing the good habit of making sure you have exactly
    the software environment you need by again issuing {\cc module purge} and
    {\cc module load gcc}. 

  \item Edit the serial code {\cc heat.c} and try to understand what it
    is doing.

    \begin{itemize}
      \item Notice how the data in the finite difference stencil is
        used, how the maximum change at each time step is evaluated
        and how the error of the numerical solution is computed at the
        end of the code.  (Starting from a sinusoidal initial
        temperature distribution and imposing constant zero
        temperature at the ends of the rod at all times, the exact
        analytic solution of the heat equation is known and the error
        can be measured exactly against this.)

      \item Compile the code with the command {\cc make heat}.  Run
        the executable using the provided input file ({\cc ./heat <
          heat.inp}) and using the submission script {\cc
          run\_heat.sh}.  Also, you can edit {\cc heat.inp} in order to
        change the parameters of the simulation (number of points
        along the rod and number of time-steps).
    \end{itemize}

  \item Edit the OpenMP code {\cc heat\_omp.c} and understand how it differs from the
    serial version in terms of loop threading.

    \begin{itemize}
      \item Compile the code with the command {\cc make heat\_omp} and
        execute the code redirecting the standard input (\ie using
        {\cc < heat.inp}) to provide the run parameters and using the
        provided job script {\cc run\_heat\_omp.sh}.

        Check the code is running correctly by comparing the output with the serial code.

      \item Identify the loop that initialises the values of the array
        {\cc u} and parallelise it using an OpenMP parallel region.

      \item Identify the loop computing the root mean square error
        between the computed and analytic solution and parallelise it
        using an OpenMP parallel region.  Is the error value computed
        close to zero as expected?  Make sure a reduce statement on
        the variable {\cc rms} exists.  (The parts in the source where
        you are asked to insert code is marked with ellipsis.)
    \end{itemize}


  \item Measure the performance of the OpenMP code
    relative to that of the serial code.  In order to do so, use a
    large number of points ($\sim$ 100k) and a large number of time
    steps ($\sim$ 10k).  Pay particular attention to the scaling,
    using 1, 2, 4, 8 and 16 processes/threads.  How do the parallel
    codes run on 1 thread compare to the serial one?  Does doubling
    the number of processes/threads double the performance? To do this 
    you will need to add the timers we used in the integral code to
    the heat code - when making these additions make sure you are measuring 
    what you want to measure. It will probably also be useful to modify
    the code to print out how many threads are being used.
    Be careful when doing this so that you don't get this being reported 
    multiple times.

\end{enumerate}

%                                                                       %
% ===================================================================== %
%                                                                       %
\end{document}

% end
